% Template: Book
% Usage: Copy this folder to start a new book project
%
% Options available:
%   - 10pt, 11pt, 12pt: font size
%   - draft, final: toggle draft mode with todo notes
%   - colorlinks, nocolorlinks: toggle colored hyperlinks
%   - singlespace, doublespace: line spacing (default: singlespace)
%   - glossary, noglossary: include glossary
%   - boxes: enable tcolorbox environments (boxcasestudy, boxconcept, boxwarning)

\documentclass[12pt, colorlinks, glossary, boxes]{bp-book}

\begin{document}

%=====================================================================
% FRONT MATTER
%=====================================================================

\frontmatter

\title{Your Book Title}
\author{Benjamin Peeters}
\date{\today}

\maketitle

%---------------------------------------------------------------------
% Abstract/Preface
%---------------------------------------------------------------------

\chapter*{Preface}
\addcontentsline{toc}{chapter}{Preface}

This is the preface to your book. It explains the motivation and scope.

%---------------------------------------------------------------------
% Table of Contents
%---------------------------------------------------------------------

\tableofcontents
\listoffigures
\listoftables

%=====================================================================
% MAIN MATTER
%=====================================================================

\mainmatter

%---------------------------------------------------------------------
% Chapter 1
%---------------------------------------------------------------------

\chapter{Introduction}
\label{chap:intro}

This is a sample book using the \texttt{bp-book} class. You can use acronyms like \gls{gdp} or \gls{imf}.

\section{Background}

Some background information here.

\section{Objectives}

The objectives of this book are:
\begin{enumerate}
    \item First objective
    \item Second objective
    \item Third objective
\end{enumerate}

%---------------------------------------------------------------------
% Chapter 2
%---------------------------------------------------------------------

\chapter{Theory}
\label{chap:theory}

\section{Theoretical Framework}

You can use theorem environments:

\begin{theorem}
This is a theorem statement.
\end{theorem}

\begin{definition}
This is a definition.
\end{definition}

\section{Case Studies and Concepts}

With the \texttt{boxes} option, you get special tcolorbox environments:

\begin{boxcasestudy}{Sample Case Study}
    This is a case study box. It's numbered within chapters and has a green frame.

    Use this for detailed examples or country studies.
\end{boxcasestudy}

\begin{boxconcept}{Sample Technical Note}
    This is a concept/technical note box. It has a blue frame.

    Use this for technical explanations or clarifications.
\end{boxconcept}

\begin{boxwarning}{Important Warning}
    This is a warning box with a red frame.

    Use this for important caveats or notes of caution.
\end{boxwarning}

%---------------------------------------------------------------------
% Chapter 3
%---------------------------------------------------------------------

\chapter{Analysis}
\label{chap:analysis}

Use \Cref{chap:intro} for smart cross-references.

Math: $\R$, $\argmin$, $\bs{x}$

\begin{equation}
    E = mc^2
    \label{eq:energy}
\end{equation}

See \Cref{eq:energy} for the equation reference.

%---------------------------------------------------------------------
% Chapter 4
%---------------------------------------------------------------------

\chapter{Conclusion}
\label{chap:conclusion}

Your conclusions here.

%=====================================================================
% BACK MATTER
%=====================================================================

\backmatter

%---------------------------------------------------------------------
% Bibliography
%---------------------------------------------------------------------

% \bibliographystyle{abbrvnat_custom}
% \bibliography{bibliographie}

%---------------------------------------------------------------------
% Glossary
%---------------------------------------------------------------------

% \printglossaries

%---------------------------------------------------------------------
% Appendices (optional)
%---------------------------------------------------------------------

% \appendix
% \chapter{Additional Data}
% Appendix content here.

\end{document}
