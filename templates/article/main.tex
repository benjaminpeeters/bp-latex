% Template: Academic Paper/Article
% Usage: Copy this folder to start a new paper project
%
% Options available:
%   - 10pt, 11pt, 12pt: font size
%   - draft, final: toggle draft mode with todo notes
%   - colorlinks, nocolorlinks: toggle colored hyperlinks
%   - singlespace, doublespace: line spacing
%   - blind: anonymous mode for submissions
%   - glossary, noglossary: include glossary
%   - onlineappendix: online appendix mode

\documentclass[12pt, colorlinks, doublespace]{bp-article}

% Bibliography path (uses $BPBIB environment variable)
% Uncomment and adjust if needed:
% \newcommand{\bibpath}{$BPBIB}

\begin{document}

%=====================================================================
% TITLE AND AUTHOR
%=====================================================================

\title{Your Paper Title}
\author{Benjamin Peeters}
\date{\today}

\maketitle

%=====================================================================
% ABSTRACT
%=====================================================================

\begin{abstract}
Your abstract goes here. This template uses the bp-article class which provides:
\begin{itemize}
    \item Automatic hyperlink coloring
    \item Double spacing (configurable)
    \item Glossary/acronym support
    \item Todo notes in draft mode
    \item Theorem environments
\end{itemize}
\end{abstract}

%=====================================================================
% MAIN CONTENT
%=====================================================================

\section{Introduction}

This is a sample paper using the \texttt{bp-article} class. You can use acronyms like \gls{gdp} or \gls{imf} which are defined in the glossary.

\section{Methods}

You can use theorem environments:

\begin{theorem}
This is a theorem statement.
\end{theorem}

\begin{proof}
This is the proof.
\end{proof}

\begin{assumption}
This is an assumption.
\end{assumption}

\section{Results}

Use \Cref{eq:sample} for smart cross-references:

\begin{equation}
    E = mc^2
    \label{eq:sample}
\end{equation}

Math commands: $\R$ for real numbers, $\argmin_{x}$, $\bs{x}$ for bold symbols.

\section{Conclusion}

Your conclusions here.

%=====================================================================
% BIBLIOGRAPHY
%=====================================================================

% Uncomment to include bibliography:
% \bibliographystyle{abbrvnat_custom}
% \bibliography{bibliographie}

%=====================================================================
% GLOSSARY (if enabled)
%=====================================================================

% Uncomment to print glossary:
% \printglossaries

\end{document}
